% Created 2021-06-11 Fri 16:09
% Intended LaTeX compiler: pdflatex
\documentclass[11pt]{article}
\usepackage[utf8]{inputenc}
\usepackage[T1]{fontenc}
\usepackage{graphicx}
\usepackage{grffile}
\usepackage{longtable}
\usepackage{wrapfig}
\usepackage{rotating}
\usepackage[normalem]{ulem}
\usepackage{amsmath}
\usepackage{textcomp}
\usepackage{amssymb}
\usepackage{capt-of}
\usepackage{hyperref}
\author{Vincent Toups}
\date{\today}
\title{}
\hypersetup{
 pdfauthor={Vincent Toups},
 pdftitle={},
 pdfkeywords={},
 pdfsubject={},
 pdfcreator={Emacs 27.2 (Org mode 9.4.4)}, 
 pdflang={English}}
\begin{document}

\tableofcontents

\section{Meta-Data Data Model}
\label{sec:orge357993}

This document describes the meta-data data model used to support the
user stories in the User Stories document.

\section{Meta-Data}
\label{sec:org75985a6}

The data model here is described in terms of JSON objects but see the
appendix for a relational description.

A meta-data object is a set of key, value pairs which we denote here
as a JSON object:

\begin{verbatim}
{ key_1:value_1, ..., key_N:value_N}
\end{verbatim}

Here \texttt{key\_N} is a string without whitespace characters.

\texttt{value\_N} denotes an array of string values without restriction on
their contents:

\begin{verbatim}
[string_1, ..., string_K]
\end{verbatim}

Eg:

\begin{verbatim}
{
    "columns":["STUDYID", "DOMAIN", "USUBJID", "RFSTDTC", "RFPENDTC",
    "BRTHDTC", "AGE", "SEX", "RACE", "RACEMULT", "ETHNIC", "STUDYID",
    "DOMAIN", "USUBJID", "QSSEQ", "QSCAT", "QSSCAT", "QSTESTCD",
    "QSTEST", "QSSTRESC", "QSSTRESN", "QSDRVFL", "VISITNUM", "VISIT",
    "QSDTC", "QSDY", "QSEVLINT", "STUDYID", "DOMAIN", "USUBJID",
    "SCSEQ", "SCTESTCD", "SCTEST", "SCMETHOD", "SCORRES", "SCORRESU",
    "SCSTRESC", "SCSTRESN", "SCSTRESU"],
    "column-formats":["text","date","integer","float"]
}
\end{verbatim}

This very simple data model is enough to support the user stories and
the query language description.
\end{document}
