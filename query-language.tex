% Created 2021-06-11 Fri 16:09
% Intended LaTeX compiler: pdflatex
\documentclass[11pt]{article}
\usepackage[utf8]{inputenc}
\usepackage[T1]{fontenc}
\usepackage{graphicx}
\usepackage{grffile}
\usepackage{longtable}
\usepackage{wrapfig}
\usepackage{rotating}
\usepackage[normalem]{ulem}
\usepackage{amsmath}
\usepackage{textcomp}
\usepackage{amssymb}
\usepackage{capt-of}
\usepackage{hyperref}
\author{Vincent Toups}
\date{\today}
\title{}
\hypersetup{
 pdfauthor={Vincent Toups},
 pdftitle={},
 pdfkeywords={},
 pdfsubject={},
 pdfcreator={Emacs 27.2 (Org mode 9.4.4)}, 
 pdflang={English}}
\begin{document}

\tableofcontents

\section{Base Query Language}
\label{sec:org8656b91}

This document describes a simple query language over the meta-data.

\subsection{Terms}
\label{sec:orgca1beea}

A <symbol> is a set of alphanumeric characters.

\begin{verbatim}
[a-ZA-Z0-9_-]+
\end{verbatim}

A <string> is a set of characters enclosed in quotation marks (with
ordinary escapes required for enclosed quotation marks.

OR a <string> is a non-quoted series of characters excluding
whitespace.

A logical operator is either:

\texttt{:or} or \texttt{:and}

A <has> is just literally \texttt{:has} and will be encoded as the latter
throughout this document. 

A <term> is either triple of the form:

\texttt{<symbol> :has <string>}

OR

<symbol> :has <string1> <logical1> <string2> <logical2>
\ldots{} <stringN>

A query is

<term1> :has <term2> [<logical1> <term3> \ldots{} <logicalN-2> <termN>]

See the reference implementation for an example of how to parse this
query language and transform it into a filter function for meta-data.
\end{document}
